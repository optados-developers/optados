\documentclass[a4paper,11pt,twoside]{book}
%\documentclass[11pt,twoside,letterpaper]{book}
\usepackage{a4wide}
%\usepackage{multirow}
\usepackage{footnote}
\usepackage{amsbsy}
\usepackage[dvips]{graphicx}
%\usepackage{fancyheadings}
\usepackage{fancyhdr}
%\setlength{\parindent}{0in}
%\setlength{\parskip}{0.05in}
\setlength{\parskip}{0.1in}

%\usepackage{a4wide}
\usepackage{color}
\def\red{\textcolor{red}}

%\parskip=2mm

% Ensure that blank pages don't have numbers or heading on them 
\makeatletter
\def\cleardoublepage{\clearpage\if@twoside \ifodd\c@page\else
 \hbox{}
 \vspace*{\fill}
 \thispagestyle{empty}
 \newpage\fi\fi}
\makeatother

% set fancy headings
\pagestyle{fancy}
\lhead[{\it \thepage}]{{\bf\it {\tt OptaDOS}: User Guide}}
\chead{}
\rhead[{\bf\it {\tt OptaDOS}: User Guide}]{{\it \thepage}}
\renewcommand{\headrulewidth}{0.2pt}
\lfoot{}
\cfoot{}
\rfoot{}
\renewcommand{\footrulewidth}{0pt}
\setlength{\footskip}{0.25in}
\setlength{\parindent}{0in}

\title{{\tt OptaDOS}: User Guide}

\author{Version 0.9}

\date{27th October 2010}

\begin{document}
\newcommand{\optados}{\texttt{OptaDOS}}

\maketitle



\chapter{Parameters}\label{chap:parameters}

\section{Usage}
{\tt
\begin{quote}
optados.x  [seedname]
\end{quote} }
\begin{itemize}
\item{ {\tt seedname}: If a seedname string is given the code will read its input
from a file {\tt seedname.odi}. The default value is {\tt castep}.}
\end{itemize}

\section{{\tt seedname.odi} File}
The \optados\ input file {\tt seedname.odi} has a flexible free-form
structure. 

The ordering of the keywords is not significant. Case is ignored (so
\verb#smearing_width# is the same as \verb#Smearing_Width#). Characters after !, or \#
are treated as comments. Most keywords have a default value that is
used unless the keyword is given in {\tt seedname.odi}. Keywords can be set
in any of the following ways
{\tt
\begin{quote}
smearing\_width = 0.4

smearing\_width : 0.4

smearing\_width   0.4
\end{quote} }
A logical keyword can be set to {\tt .true.} using any of the following
strings: {\tt T}, {\tt true}, {\tt .true.}.


\clearpage


\section{Parameters}
\subsection[task]{\tt character(len=50) :: task}

Tells the code what to compute. 

The valid options for this parameter are:
\begin{itemize}
\item[{\bf --}]  \verb#dos# (default)
\item[{\bf --}]  \verb#jdos#
\item[{\bf --}]  \verb#pdos#
\item[{\bf --}]  \verb#optics#
\item[{\bf --}]  \verb#core#
\item[{\bf --}]  \verb#all#
\end{itemize}
Several taks can be specified eg to compute dos and jdos use
\verb#task : dos jdos#.

\subsection[broadening]{\tt character(len=50) :: broadening}

Specified the scheme used to broaden a discrete sampling of the
Brillouin Zone to a continuous spectral function.

The valid options for this parameter are:
\begin{itemize}
\item[{\bf --}]  \verb#adaptive# (default)
\item[{\bf --}]  \verb#fixed#
\item[{\bf --}]  \verb#linear#
\item[{\bf --}]  \verb#quad#
\end{itemize}
Several schemes can be specified eg 
\verb#broadening : adaptive fixed#.

\subsection[iprint]{\tt integer :: iprint}

This indicates the level of verbosity of the output from 1,
the bare minimum, to 3, which corresponds to full debugging output.

The default value is 1.


\subsection[length\_unit]{\tt character(len=20) :: energy\_unit}
The energy unit to be used for writing quantities in the output files.

The valid options for this parameter are:
\begin{itemize}
\item[{\bf --}]  \verb#eV# (default)
\item[{\bf --}]  \verb#Ry#
\item[{\bf --}]  \verb#Ha#
\end{itemize}

\subsection[adaptive\_smearing]{\tt real(kind=dp) :: adaptive\_smearing}
Set the relative smearing in the adaptive scheme

Default value is 0.4

\subsection[fixed\_smearing]{\tt real(kind=dp) :: fixed\_smearing}
Smearing width for fixed broadening

If $\verb#spectral_scheme# = \verb#fixed#$ default value is 0.3eV.

\subsection[scissor\_op]{\tt real(kind=dp) :: scissor\_op}
Value of the scissor operator. 

Default value is 0eV (ie not used)

\subsection[compute\_efermi]{{\tt logical :: compute\_efermi}}

If {\tt compute\_efermi=TRUE}, then \optados\ will use the value of the fermi
level computed from the integration of the DOS. If {\tt
  compute\_efermi=FALSE}, then the value set by {\tt fermi\_energy} will be
used, if {\tt fermi\_energy} is not set the value from the bands file will be used.

The default value is {\tt FALSE}.

\subsection[fermi\_energy]{\tt real(kind=dp) :: fermi\_energy}
Value of the fermi energy

No default value

\subsection[output\_format]{\tt character(len=20) :: output\_format}
Format in which to output data

The valid options for this parameter are:
\begin{itemize}
\item[{\bf --}]  \verb#gnuplot#
\item[{\bf --}]  \verb#grace# (default)
\end{itemize}

\subsection[finite\_bin\_correction]{\tt logical :: finite\_bin\_correction}


\subsection[set\_efermi\_zero]{\tt logical :: set\_efermi\_zero}


\subsection[dos\_min\_energy]{\tt real(kind=dp) :: dos\_min\_energy}
Lower energy range for DOS and related properties.

Default value is 5eV below the lowest eigenvalue in the bands file.

\subsection[dos\_max\_energy]{\tt real(kind=dp) :: dos\_max\_energy}
Upper energy range for DOS and related properties.

Default value is 5eV above the highest eigenvalue in the bands file.

\subsection[dos\_spacing]{\tt real(kind=dp) :: dos\_spacing}
Resolution at which to compute the DOS and related properties.
Default value is 0.1eV (AJM??)

\subsection[jdos\_max\_energy]{\tt real(kind=dp) :: jdos\_max\_energy}
Upper energy range for JDOS and related properties.

Default value is the difference between the valence band maximum (or
fermi level) and the highest eigenvalue in the bands file.

\subsection[jdos\_spacing]{\tt real(kind=dp) :: jdos\_spacing} 
Resolution at which to compute the DOS and related properties.
Default value is 0.1eV (AJM??)

\subsection[optics\_geom]{\tt character(len=20) :: optics\_geom}

\begin{itemize}
\item[{\bf --}]  \verb#polycrystalline# (Isotropic average)
\item[{\bf --}]  \verb#polarized#  
\item[{\bf --}]  \verb#unpolarized# 
\item[{\bf --}]  \verb#tensor# (Full dielectric tensor)
\end{itemize}


\subsection[optics\_qdir]{\tt real(kind=dp) :: optics\_qdir(3)}
Direction of polarisation. Must be specified if \verb#optics_geom :polarized#  
or \verb#optics_geom : unpolarized# .
No default


\subsection[pdos]{\tt integer :: pdos}
Defines which components to include in the pdos analysis:

\begin{itemize}
\item[{\bf --}]  \verb#angular# (decompose as s,p,d etc)
\item[{\bf --}]  \verb#atom#    (decompose onto atomic sites)
\item[{\bf --}]  \verb#C:H#     (decompose onto Carbon and Hydrogen sites)
\item[{\bf --}]  \verb#C1:C3:C4-C8#  (decompose onto atoms C1, C2 and C4,C5,C6,C7,C8)
\item[{\bf --}]  \verb#Si1[s;d]#     (decompose onto 's' and 'd' channels for
  atom Si1)

\end{itemize}





\subsection[devel\_flag]{\tt character(len=50) :: devel\_flag}

Not a regular keyword. Its purpose is to allow a developer to pass a
string into the code to be used inside a new routine as it is developed.

No default.




\end{document}
